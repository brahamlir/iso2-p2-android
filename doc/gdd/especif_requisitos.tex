
% -*- coding:utf-8 -*-
\section{Especificación de Requisitos}

\subsection{Introducción}
\subsubsection{Propósito}
El propósito de este documento es servir de soporte a la
especificación de requisitos en este primer contacto del proyecto. Se
explicarán las razones por la cual se considera más que viable y
rentable el proyecto. Se identificarán a los implicados por parte de
la empresa a la cual está destinada el producto (videojuego). Se
explicarán las características, requisitos y restricciones del
proyecto.

El proyecto consta de un videojuego COMPLETAR.....

\subsubsection{Alcance}
El alcance del documento es una visión general del proyecto, reuniendo
las características principales. Se identificarán nuevos requisitos en
mayor detallen en siguientes versiones. No obstante, la visión que se
procurará dar será lo suficientemente completa y precisa como para
tener una noción concreta del proyecto a desarrollar.


\subsection{Descripción general}
\begin{itemize}
\item Oportunidad de negocio
  COMPLETAR
\item Sentencia que define el problema
COMPLETAR
\item Sentencia que define la posición del Producto
COMPLETAR
\item Descripción de Stakeholders (Participantes en el Proyecto)
  y Usuarios
  LO MÁS PROBABLE ES QUE NO SE NECESITE COMPROBAR
\item Descripción Global del Producto
  Perspectiva del producto COMPLETAR
  Resumen de características COMPLETAR

\end{itemize}

\chapter{Requisitos específicos}
\begin{itemize}
\item Resumen de requisitos generales
\item Descripción Global del Producto
EJEMPLO DE LA PRIMERA ENTREGA:

\begin{itemize}
\item Consultar ayuda
\item Acceder a opciones
\item Ver puntuaciones
\item Jugar
\end{itemize}

\item Restricciones
  COMPLETAR
\item Otros Requisitos del Producto
  COMPLETAR
\end{itemize}

