% -*- coding:utf-8 -*-
\section{Presentación de la empresa}

{\bf Shurdroid S.L.} es una empresa orientada al desarrollo de videojuegos.
Como punto de partida desarrolla videojuegos para plataformas móviles, en
primera instancia para dispositivos con sistema operativo Android.

Los objetivos a largo plazo son adquirir experiencia y prestigio en el mundo de
los videojuegos para poder abordar proyectos de mayor envergadura.

\subsection{Organigrama de la empresa}

La empresa está compuesta por un coordinador y tres desarrolladores. El modelo
organizativo es lo más horizontal posible, ya que somos un equipo pequeño. El
coordinador también ejerce como desarrollador cuando no está ocupado con tareas
organizativas.

El organigrama queda de la siguiente forma:

\begin{itemize}

\item {\bf Coordinador:} Manuel José Abaldea

\item {\bf Desarrolladores:}

\begin{itemize}
\item Luis Miguel García-Muñoz
\item Eduardo Monroy
\item Felipe Terriza
\end{itemize}

\end{itemize}

\subsection{Parque de recursos}

La mayoría del trabajo que realizamos se hace desde casa. Para ello, la empresa
proporciona a cada miembro los siguients recursos:

\begin{itemize}
\item Ordenador de sobremesa (Intel i3 2 GHz, 4 GiB memoria ram, ...), pantalla
23'', teclado y ratón incluidos. Valorado en 1000 euros.
\item Conexión a internet ADSL proporcionada por Vodriofone (6 Mb bajada, 3 Mb
subida), por 30 euros al mes.
\end{itemize}

No se incluye coste del software porque sólo se usará software libre y
gratuito. Las reuniones presenciales se realizarán en casa del coordinador,
para las cuales la empresa pagará los gastos de desplazamiento. Todos los
empleados, coordinador incluido, cobran un salario de 18000 euros brutos al
año, distribuidos en 12 pagas.

\section{Proyecto propuesto}

Videojuego 2D tipo plataformas/rpg (todavía por decidir) para plataforma
Android. El producto se venderá en el {\em market} de Android a una cantidad de
$0.90$ euros la descarga.

\subsection{Duración y gastos previstos del proyecto}

El proyecto está previsto terminarlo en 4 meses, por lo que los gastos serán:

\begin{itemize}
\item Coste inicial del equipamiento: 4000 euros
\item Gasto asociado a 4 meses de internet: 480 euros
\item Salarios de todos los miembros para los 4 meses: 18000 euros
\item Gastos en desplazamientos (estimada 1 reunión presencial semanal): 480 euros
\end{itemize}

Por tanto, se estima un coste total del proyecto de 22960 euros. Se considera
además un colchón de 5000 euros para posibles gastos no previstos.

\subsection{Previsión de ingresos}

Para recuperar el coste del proyecto, se requerirán entre 20664 y 25164
descargas (dependiendo de cuantos gastos no previstos hayan surgido)

\section{Entorno}

\subsection{Entorno tecnológico}

\begin{itemize}

\item Las comunicaciones en el grupo se realizarán mediante el correspondiente grupo
habilitado en {\em Google Groups} y también mediante {\em Google Talk}.

\item Control de versiones de código: repositorio mercurial en {\em Google Code}.

\item Herramientas de desarrollo: editor preferido por cada desarrollador y SDK
de {\em Android}

\item Herramientas de diseño UML: Umbrello

\end{itemize}

\subsection{Entorno metodológico}

Se plantea el uso del Proceso Unificado de Rational como metodología
de desarrollo, pero se opta por no seguir al 100\% cada uno de sus
apartados, ya que a pesar de ser un método muy completo no está
enfocado al desarrollo de videojuegos. Genera demasiada documentación
que conlleva a un mayor gasto de tiempo, dinero y personal.

Dentro del PUD adaptado se utilizará Stream programing o Scrum como
proceso de desarrollo.
