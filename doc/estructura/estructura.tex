% -*- coding:utf-8 -*-
\section{Estructura de la empresa}
\subsection{Presentación de la empresa}
Nombre: SHURDROID

Shurdroid es una empresa orientada al desarrollo de videojuegos. Como
punto de partida desarrolla videojuegos para plataformas móviles, en
primera instancia para smartphones con sistema operativo Android.

Los objetivos a largo plazo son adquirir experiencia y prestigio en el
mundo de los vídeo-juegos para poder abordar proyectos de mayor
envergadura.


\subsubsection{Organigrama de la empresa}
La empresa está compuesta por un Analista, un diseñador y dos
programadores/tester.

Estos cuatro componentes son los fundadores de la empresa. Los cargos
se distribuyeron según experiencia laboral y acuerdo entre los
fundadores.

\begin{itemize}
\item \underline{Coordinador/Analista}:

Su labor es generar una visión inicial del producto, análisis de
requisitos y funcionalidades del software a desarrollar, estudio y
aplicación de la retroalimentación generada por el desarrollo.
\item \underline{Diseñador}:

Basándose el en análisis, debe evaluar las posibles soluciones y optar
por las más adecuadas para la empresa y como su propio rol indica,
diseñar una solución apropiada.

\item \underline{Programadores/Testers}:

Basándose en el diseño del producto deben implementar una aplicación
funcional con los requisitos y fines dados.
\end{itemize}

\subsubsection{Parque de recursos}
La sede de la empresa consta de:
\begin{itemize}
\item Despacho de dirección.
\item Sala de juntas.
\item Laboratorio de trabajo.
\item Aseo.
\end{itemize}

Los recursos con los que cuenta la empresa son:
\begin{itemize}
\item Un personales de uso genérico (Intel i5, 4GB ram)
  a cargo del Analista en el despacho de dirección
\item Un portátil de uso genérico  (Intel i5, 4GB ram) a cargo del
  Diseñador.
\item Dos portátiles orientados a desarrollo (Intel i7, 8GB ram,
  gráfica dedicada 2GB) a cargo de los testers.
\item Mini-servidor HP ProLiant ML110 para backup de archivos.
\end{itemize}

Se ha optado por los equipos portátiles para dar la opción a los
empleados al teletrabajo.

Los recursos software se orientarán al uso de software libre, tanto
por el ahorro de licencias como por la afinidad a los principios del
software libre.

\subsubsection{Previsión de gastos de inversión}
Inversión inicial de la empresa:
\begin{itemize}
\item Equipos informáticos:      4300€
\item Muebles material fungible: 3800€
\end{itemize}

Media de gastos fijos:
\begin{itemize}
\item Alquiler 500€/mes
\item Luz       60€/mes
\item Agua      15€/mes
\item Telefonía 90€/mes
\item Internet  50€/mes
\end{itemize}

Coste de personal(incluye seguros e impuestos):
\begin{itemize}
\item Analista    90€/h
\item Diseñador   60€/h
\item Programador 30€/h
\item Limpieza   650€/mes (contratación a media jornada)
\end{itemize}

Media de gastos mensuales fijos (incluida limpieza): 1365€/mes
Gastos de personal (según horas asignadas al proyecto)


\subsection{Proyecto propuesto}
Videojuego 2D tipo puzzle para plataforma Android.
El producto se asociará en el market de Android a 0,90€ la
descarga. El código del mismo será liberado al alcanzar una
amortización mínima.

DURACIÓN Y GASTOS PREVISTOS DEL PROYECTO
Analista        160h => 14400€
Diseñador       240h => 14400€
ProgramadorX2   360h => 21600€
Gastos fijos  2meses =>  2730€

Total 53.130€


PREVISIÓN DE INGRESOS
Previsión de 2000 descargas => 1800€

RESUMEN:
Pérdidas de 50330€

\subsection{Entorno tecnológico}

- Actas y reuniones: con google-groups y en chat.
- Control de versiones de código: repositorio mercurial en google-code
- Herramientas de desarrollo: emacs, vi, eclipse, sdk plattform
android,  ADV manager.
- Herramientas para diseño UML: Umbrello (Software libre)
- Herramienta de planificación: Openproj


\subsection{Entorno metodológico}
Se plantea el uso del Proceso Unificado de Rational como metodología
de desarrollo, pero se opta por no seguir al 100\% cada uno de sus
apartados, ya que a pesar de ser un método muy completo no está
enfocado al desarrollo de videojuegos. Genera demasiada documentación
que conlleva a un mayor gasto de tiempo, dinero y personal.

Dentro del PUD adaptado se utilizará Stream programing o Scrum como
proceso de desarrollo.